\SetPicSubDir{my-background}

\chapter{Background}
\label{ch:background}

\vspace{2em}


\section{Properties of the memristor \ac{CBA}}

To perform a multiplication in the analog domain, Ohm's Law is taken advantage of. By applying a voltage across a resistive element $R$, the current $I$ is a multiplication of the voltage $V$ and the conductance $G$, where $G = \sfrac{1}{R}$.

  $$
    VG = I
  $$

In order to utilise this to do computing tasks, a precise way to program the resistance on-the-fly is desired. A promising device is the memristor which has the ability to retain its resistance state \cite{Strukov_Snider_Stewart_Williams_2008}. Its resistance state can be changed by applying a set or reset threshold voltage pulse. This is explained later on with its IV characteristic curve in \autoref{background:fig:memristor_iv_curve}. Furthermore, it is a passive two-terminal device which uses very low amounts of power in comparison to transistors or other active devices \cite{Strukov_Snider_Stewart_Williams_2008}.

\section{Matrix operations on the \ac{CBA}}
\label{sec:matrix_operations_on_the_cba}

  \begin{figure}[ht]
    \centering
    \includegraphics[width=1.0\linewidth]{\Pic{jpg}{gr1_lrg}}
    \caption{Simplified physical structure of the \acl{CBA} and its schematic diagram \cite{Joksas_Mehonic_2020}}
    \label{background:fig:crossbar}
  \end{figure}

Subsequently, joining parallel current lines performs a summation of the currents in accordance to \ac{KCL}. With both multiplication and addition, a \ac{MAC} operation has been created. \ac{MAC} operations are used extensively in neural network applications. Scaling up, the memristors are placed into a \ac{CBA} like shown in \autoref{background:fig:crossbar}. A matrix to be stored in the memory is mapped to the conductance states of the memristor crossbar array. The rows are called the wordlines and the columns are called the bitlines. 

\section{Forward and transpose \ac{MAC} operation}
\label{sec:forward_and_transpose_mac_operations}

By convention, the wordline is typically driven with a voltage. With the example of a $3\times5$ \ac{CBA} as in \autoref{background:fig:crossbar}, the forward \ac{MAC} operation is performed where a vector of voltages is applied at the wordline, and a current sense at the bitline will obtain the output results \cite{Ielmini_2016}. 

  $$
  V_{1\times3} \times G_{3\times5} = I_{1\times5}
  $$

However, if the bitline is driven with a voltage instead, another set of results will be obtained at the wordline current sense. A \ac{MAC} operation with $G^T$, the transpose of the memory $G$, is realised.

  $$
  V_{1\times5} \times \left(G^{T}\right)_{5\times3} = I_{1\times3}
  $$

For \ac{ANN} applications, the forward \ac{MAC} is used extensively in inference with in-memory computation is performed on the \ac{CBA} using the weight matrix stored in the memory. With the transpose \ac{MAC} operation, it is also possible to accelerate \ac{ANN} training algorithms such as backpropagation and gradient descent \cite{Crafton_West_Basnet_Vogel_Raychowdhury_2019}.

\section{Considerations when using \acp{CBA} for \ac{ANN} operations}
\label{sec:considerations_cba_for_ann}

This section outlines the key considerations when using memristor-based hardware for neural network operations.

\begin{figure}[ht]
  \centering
  \includegraphics[width=1.0\linewidth]{\Pic{png}{memristor_iv_curve}}
  \caption{IV characteristic curve of a memristive device \cite{CAMPBELL201710}. Measurements of the high and low resistive states are added to the figure.}
  \label{background:fig:memristor_iv_curve}
\end{figure}

Firstly in an \ac{ANN} model, weights are typically normalised and stored as floating points between -1 to +1. The first issue is that memristors can only have positive conductances, hence we must come up with a method for computations involving negative weights. 

If we consider only the positive range of 0 to +1, an ideal memristor should have a programmable conductance ranging from 0 $\mho$ and 1 $\mho$ (or a resistive state ranging from $\infty$ $\Omega$ to $1$ $\Omega$). A practical memristor has a finite high resistive state and a large low resistive state. As seen in \autoref{background:fig:memristor_iv_curve}, \citet{CAMPBELL201710} \cite{CAMPBELL201710} achieves a ratio $\sfrac{R_{HRS}}{R_{LRS}}$ of only $\sfrac{1200 \Omega}{200 \Omega} = 6$.

Multiple \acp{CBA} will also have to be cascaded together. This maps to the multiple layers in \acp{ANN}. Since the result from the first \ac{CBA} is current, a current-to-voltage converter is needed to create the voltage to pass into the second \ac{CBA}. Additionally, interfacing circuits for activation functions need to be designed as well.

Another major difficulty is the ability to achieve intermediate states between the high and low resistance states. The intermediate states are unstable; there may be drifts in the conductances as results are being read continuously \cite{Ielmini_2016}.
Therefore, this project shall work with the constraint that each memristor can only achieve two states reliably. In other words, each device can only be used for only 1-bit binary representation. In particular, a \ac{BNN} will be used as a proof-of-concept for mapping of the hardware architecture.

With the above issues in mind, the possible solutions will be addressed in \fullref{ch:methodology}.
