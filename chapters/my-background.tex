\SetPicSubDir{my-background}

\chapter{Background}
\label{ch:background}

\vspace{2em}


\section{Properties of the memristor \ac{CBA}}

To perform a multiplication in the analog domain, Ohm's Law is taken advantage of. By applying a voltage across a resistive element $R$, the current $I$ is a multiplication of the voltage $V$ and the conductance $G$, where $G = R^{-1}$.

  $$
    VG = I
  $$

In order to utilise this up to do computing tasks, a precise way of being able to program the resistance on the fly is desired. A promising device is the memristor which has the ability to retain its resistance state \cite{Strukov_Snider_Stewart_Williams_2008}. Its resistance state can be changed by applying a set or reset threshold voltage pulse. Furthermore, it is a passive two-terminal device which uses very little power in comparison to transistors or other active devices \cite{Strukov_Snider_Stewart_Williams_2008}.

\section{Matrix operations on the \ac{CBA}}
\label{sec:matrix_operations_on_the_cba}

  \begin{figure}[ht]
    \centering
    \includegraphics[width=1.0\linewidth]{\Pic{jpg}{gr1_lrg}}
    \caption{Simplified physical structure of the \acl{CBA} and its schematic diagram \cite{Joksas_Mehonic_2020}}
    \label{background:fig:crossbar}
  \end{figure}

Subsequently, joining parallel current lines performs a summation of the currents in accordance to \ac{KCL}. With both multiplication and addition, a \ac{MAC} operation has been created. \ac{MAC} operations are used extensively in neural network applications. Scaling up, the memristors are placed into a \ac{CBA} like shown in \autoref{background:fig:crossbar}. A matrix to be stored in the memory is mapped to the conductance states of the memristor crossbar array. The rows are called the wordlines and the columns are called the bitlines. 

\section{Forward and transpose \ac{MAC} operation}
\label{sec:forward_and_transpose_mac_operations}

By convention, the wordline is typically driven with a voltage. With the example of a $3\times5$ \ac{CBA} as in \autoref{background:fig:crossbar}, the forward \ac{MAC} operation is performed where a vector of voltages is applied at the wordline, and a current sense at the bitline will obtain the output results \cite{Ielmini_2016}. 

  $$
  V_{1\times3} \times G_{3\times5} = I_{1\times5}
  $$

However, if the bitline is driven with a voltage instead, another set of results will be obtained at the wordline current sense. A \ac{MAC} operation with $G^T$, the transpose of the memory $G$, is realised.

  $$
  V_{1\times5} \times \left(G^{T}\right)_{5\times3} = I_{1\times3}
  $$

For \ac{ANN} application, the forward \ac{MAC} is used extensively in inference and in-memory computation is performed on the \ac{CBA} using the weight matrix stored in the memory. It is also possible to accelerate \ac{ANN} training algorithms such as backpropagation and gradient descent which require transpose \ac{MAC} operations \cite{Crafton_West_Basnet_Vogel_Raychowdhury_2019}.

\section{Negative number representation}

Since the memristor conductance can only be positive, the matrix that is stored in the physical \ac{CBA} memory must only be positive. A method to map a negative number is to split up the number across two memristors such that $G = G_{A} - G_{B}$, where the device conductances $G_{A}$ and $G_{B}$ are positive values \cite{Sun_Ielmini_2022}. Two types of splitting methods can be used and \autoref{background:fig:col_row_splitting} depicts that under a wordline voltage drive.
\\

  \begin{figure}[ht]
    \centering
    \includegraphics[width=1.0\linewidth]{\Pic{png}{col_row_splitting}}
    \caption{Adjacent splitting methods in a \ac{CBA}. (a) Column-wise splitting. (b) Row-wise splitting.}
    \label{background:fig:col_row_splitting}
  \end{figure}

\autoref{background:fig:col_row_splitting}(a) performs column-wise splitting where single row of wordline voltage drive $V_1$ results in two columns of currents where the result can be retrieved with a subtraction circuit $I_1 = I_{1A} - I_{1B}$. \autoref{background:fig:col_row_splitting}(b) performs a row-wise splitting where two rows of wordline voltage drives, $V_1$ and $V_2 = - V_1$, creates a single column of current $I_1$ by law of \ac{KCL}.

% Overall, the pros and cons are detailed in this table:

%   \begin{table}[h]
%     \centering
%     \begin{tabular}{|c||c|c|}
%       \hline
%       \text{} & Pros & Cons \\ 
%       \hline \hline
%       MainDoor & PasswordCheck & MainDoor \\ 
%       \hline
%       PasswordCheck & MainHall & MainDoor \\ 
%       \hline
%     \end{tabular} 
%     \caption{Summary of \ac{CBA} splitting techniques}
%   \end{table}



% TODO Matrix Inverse
%https://www.pnas.org/doi/10.1073/pnas.1815682116
% https://ieeexplore-ieee-org.libproxy1.nus.edu.sg/document/9774858

