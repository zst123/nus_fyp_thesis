\SetPicSubDir{my-framework}

\chapter{Framework}
\label{ch:framework}

\vspace{2em}

\section{Abstraction Layer \& Simulation Backend}

With the proposed circuits in \autoref{ch:methodology} \nameref{ch:methodology}, the functionality of the entire system will be validated by building up a simulation backend. The backend is abstracted in a Python package to closely resemble the functionality of the actual \ac{CBA}. As illustrated in \autoref{framework:fig:abstraction_layers}, the aim is to allow for the low-level abstraction layer to maintain interoperability when redeploying the code on future developments of the hardware without affecting the high-level software stack. The simulation \acp{API} are designed to be easily swapped with the actual hardware \acp{API} and run interoperably with the algorithms at higher levels.

\begin{figure}[!ht]
  \centering
  \includegraphics[width=0.95\linewidth]{\Pic{png}{abstraction_layers_ca2}}
  \caption{Abstraction layer for the actual hardware (left) and the simulation backend (right)}
  \label{framework:fig:abstraction_layers}
\end{figure}

\section{Generator-Simulator Stack}

\begin{figure}[!ht]
  \centering
  \includegraphics[width=0.8\linewidth]{\Pic{png}{driver_stack_ca2}}
  \caption{Python functions in the generator\textendash{}stack}
  \label{framework:fig:driver_stack}
\end{figure}

At the driver level, the netlists for the \ac{CBA} can be auto-generated using the Python generation stack by passing in parameters shown in \autoref{framework:fig:driver_stack}. Configurable parameters include the size of the crossbar (the number of wordlines and bitlines) and a list of resistance states (supporting both binary and intermediate resistance states).
The SPICE interface driver is designed to be cross-compatible with LTspice and Hspice. More details of the cross-compatible setup are explained in \fullref{appendix:spice}.

\begin{figure}[!ht]
\begin{minipage}[b]{0.49\textwidth}
  \centering
  \includegraphics[width=\linewidth]{\Pic{png}{subcircuit_cell}}
  \caption{Generated cell subcircuit}
  \label{framework:fig:subcircuit_cell}
\end{minipage}%
\begin{minipage}[b]{0.49\textwidth}
  \centering
  \includegraphics[width=\linewidth]{\Pic{png}{subcircuit_cba}}
  \caption{Generated \ac{CBA} subcircuit}
  \label{framework:fig:subcircuit_cba}
\end{minipage}%
\end{figure}

\newpage

\autoref{framework:fig:subcircuit_cell} and \autoref{framework:fig:subcircuit_cba} are examples of the generated subcircuits. The subcircuit for the memristor is currently modelled in a 1R (one-resistor) configuration and can be expanded easily to other common configurations such as 1D1R (one-diode \& one-resistor) or 1T1R (one-transistor \&  one-resistor) by editing the subcircuit. Multiple cells are combined to form the \acl{CBA} subcircuit.
For deployment, the following Python \ac{API} functions are implemented as in \autoref{framework:fig:cal_layer_python}

\iffalse
\section{\Ac{CAL}}

\begin{figure}[!ht]
  \centering
  \includegraphics[width=1.0\linewidth]{\Pic{png}{cal_layer_encap_ca2}}
  \caption{Multiple \acp{CBA} combined into a single runtime}
  \label{framework:fig:cal_layer_encap_ca2}
\end{figure}

To support the dual \ac{CBA} configuration in \autoref{sec:methodology:negative_number_representation} \nameref{sec:methodology:negative_number_representation}, the \acf{CAL} exposes a high level runtime layer which can incorporate multiple physical \acp{CBA} within it (see \autoref{framework:fig:cal_layer_encap_ca2}).
Hence, many of these small \acp{CBA} may be combined and abstracted into a runtime. This is especially so since multiple small \acp{CBA} are easier to manufacture with a higher yield, and it is very likely that combining \acp{CBA} will be physically needed.

The user may use the runtime as if it were a single large \ac{CBA}. The Python \ac{API} functions are outlined in \autoref{framework:fig:cal_layer_python}.
%This functionality is demonstrated in \fullref{ch:results}.
\fi

\begin{figure}[!ht]
  \centering
  \includegraphics[width=1.0\linewidth]{\Pic{png}{cal_layer_python_ca2}}
  \caption{Python functions in the Crossbar Abstraction Layer}
  \label{framework:fig:cal_layer_python}
\end{figure}

%% Sem 2
\section{Generation of neural network circuitry}

\noindent
With the subcircuits generated above, the script can also cascade several layers together. \autoref{framework:fig:generated_cba:4x2x3} shows a close up of the initial generated schematic. It consists of two sets (\acf{CBA} and neuron circuit) cascaded together. After the functionality has been verified, a larger network (consisting of 100-inputs, 50-hidden, and 10-output nodes) was generated for the demonstration of a MNIST model network (see \autoref{framework:fig:generated_cba:100x50x10}). The performance of the hardware neural network will be characterised in \fullref{ch:results}.

\begin{figure}[!ht]
  \centering

  \begin{subfigure}{0.85\textwidth}
    \includegraphics[width=\linewidth]{\Pic{png}{generated_4x2x3}}
    \caption{Network of $4\times 2\times 3$ \textendash{} for testing purposes}
    \label{framework:fig:generated_cba:4x2x3}
  \end{subfigure}

  \vspace{2em}

  \begin{subfigure}{1.0\textwidth}
    \includegraphics[width=\linewidth]{\Pic{png}{generated_100x50x10}}
    \caption{Network of $100\times 50\times 10$ \textendash{} for MNIST model demo}
    \label{framework:fig:generated_cba:100x50x10}
  \end{subfigure}

  \caption{Generated neural network circuit diagrams by the generator\textendash{}stack}
  \label{framework:fig:generated_cba}
\end{figure}
