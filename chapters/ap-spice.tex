\SetPicSubDir{appendix}

\chapter{Comparing LTspice \& Hspice}
\label{appendix:spice}

%%
\section{Overview of SPICE environment}

In \autoref{ch:framework} \nameref{ch:framework}, the simulation backend is designed to be cross-compatible with both LTspice and Hspice.
Hspice is a commercial software and is extremely fast; however, the lack of a graphical interface is a barrier to rapid prototyping for the whole team. LTspice is a free software and is commonly taught in undergraduate curricula due to its graphical interface and ease of use.

With this, it was decided that being able to generate cross-compatible netlists for software interoperability will be advantageous to address the initial phase of prototyping and debugging, before scaling up into large batch simulations.
This appendix aims to inform the reader on the key differences between the two and note the cross-compatibility features built into the simulation backend.

%%
\section{Differences in SPICE netlist syntax}

\begin{figure}[ht]
    \centering
    \includegraphics[width=1.0\linewidth]{\Pic{png}{spice_syntax_differences}}
    \caption{Average time taken per simulation}
    \label{appendix:fig:spice_syntax_differences}
\end{figure}

Both LTSpice and HSpice netlists have non-standard SPICE syntax. \autoref{appendix:fig:spice_syntax_differences} highlights the LTspice-specific syntax in red, and Hspice-specific syntax in green.
\\

\begin{description}[noitemsep,nolistsep]
    \item[Passing parameters] The indicator {\verb|params: |} can be omitted to make it cross-compatible.
    \item[Parameter substitution] Parameters can be referenced using quotations {\verb|''|} instead of braces {\verb|{}|} to make it cross-compatible.
    \item[Simulation settings] LTspice requires {\verb|.backanno|} to run the simulation, while Hspice requires {\verb|.OPTIONS POST=2 nomod|} to print out the output data from the simulation into a file.
\end{description}

%%
\section{Performance benchmarks}

\begin{table}[h]
    \centering
    \begin{minipage}{.4\linewidth}
        \begin{tabular}{|c|c|}
            \hline
            \multicolumn{2}{|c|}{\textbf{100 simulations in...}} \\
            \hline
            LTSpice (sec) & HSpice (sec) \\ 
            \hline
            \hline
            107.056 & 26.626 \\ 
            \hline
            106.157 & 25.938 \\ 
            \hline
            107.213 & 26.143 \\ 
            \hline
            106.804 & 26.002 \\ 
            \hline
            107.329 & 26.489 \\
            \hline
        \end{tabular} 
    \end{minipage}%
    \begin{minipage}{.5\linewidth}
        \begin{tabular}{|c|c||c|}
            \hline
            \multicolumn{3}{|c|}{\textbf{Average time taken per simulation}} \\
            \hline
            LTSpice (sec) & HSpice (sec) & \hl{Speed-up} \\ 
            \hline
            \hline
            1.069118 & 0.262396 & \hl{$4.07 \times$}\\ 
            \hline
        \end{tabular} 
    \end{minipage} 
    \caption{Sample time taken to run simulations in each environment}
    \label{appendix:table:spice_benchmark}
\end{table}

\begin{figure}[ht]
    \centering
    \includegraphics[width=0.8\linewidth]{\Pic{png}{spice_benchmark}}
    \caption{Average time taken per simulation}
    \label{appendix:fig:spice_benchmark}
\end{figure}

\noindent The benchmark results are tabulated in \autoref{appendix:table:spice_benchmark}. As seen in \autoref{appendix:fig:spice_benchmark}, Hspice is $4.07 \times$ faster than LTspice for batch simulation.

\text{}

\noindent The following parameters were set for both instances for the benchmark:

\begin{itemize}[noitemsep,nolistsep]
    \item    Simulation with a single instance and enabling multi-threads
    \item    Circuit used is the $8 \times 8$ crossbar with hardcoded voltage inputs.
    \item    Full transient analysis was simulated for 10 milliseconds
    \item    Running in headless mode for batch simulation 
\end{itemize}

\text{}

\noindent The main overhead in LTspice is the simulation itself. While it supports multi-threaded simulations, it does not use up the maximum resources of the \ac{CPU}. On the contrary, Hspice launches more processes to perform more computations concurrently. 
As Hspice is a paid software, the main overhead when launching Hspice is the authentication with the license server. Hence, it is recommended that latency from the workstation to the license server is minimised. 

%%
