\section{Noise sensitivity analysis}
\label{results:noise_sensitivity_analysis}

\subsection{Implementation of noisy \ac{BNN} model}
\label{results:impl_of_noisy_bnn_model}

\noindent
Firstly, a MNIST classifier model is trained on a $100 \times 50 \times 10$ \ac{BNN}. The noisy distribution characterised in \autoref{results:noisemodel} is cascaded after the \ac{CBA} weights.

\begin{figure}[!ht]
    \centering
    \includegraphics[width=1.0\linewidth]{\Pic{png}{noisy_bnn_model}}
    \caption{Noisy \ac{BNN} architecture of $100 \times 50 \times 10$. The noisy layers are cascaded after the weights}
    \label{results:fig:trisurf_1}
\end{figure}

\noindent
After which, noisy layers are introduced to observe how the test accuracy drops. The drop in feedforward accuracy is observed by varying parameters in the noisy layers. \autoref{results:fig:trisurf_1} shows the relationship between accuracy against the two parameters that affect the crossbar noise distribution.

\vspace{0.5em}
\setlength{\fboxsep}{1em}
\noindent
\framebox[\textwidth][l]{\parbox{0.9\textwidth}{
    \underline{\textbf{Observations:}}

    With the CBA programmed with the weights obtained from offline (or, off-chip) training:

    \begin{enumerate}
        \item Accuracy decreases as cell-level standard deviation (C) increases.
        \item Accuracy decreases as the ratio of the high and low resistance state($\sfrac{R_{HRS}}{R_{LRS}}$) decreases.
    \end{enumerate}
}}

\vspace{1em}

\noindent
Using \autoref{results:fig:trisurf_2}, we can choose the the maximum allowable loss in accuracy for the application and predict the required parameters for the system.

\vspace{0.7em}

\noindent
Consider an allowable accuracy loss of 2\%, with a memristor device of $R_{HRS} = 11k\Omega$ and $R_{LRS} = 500\Omega$, the ratio is 22. From \autoref{results:fig:trisurf_2}, the allowable standard deviation is less than 20\%. Hence the designers must adhere to this constraint to maintain acceptable levels of performance.
\vspace{0.7em}

\begin{figure}[!ht]
    \centering
    \includegraphics[width=0.85\linewidth]{\Pic{png}{trisurf_1}}
    \caption{3D Plot showing the relationship between test error, resistance state ratio and cell variations}
    \label{results:fig:trisurf_1}
\end{figure}

\begin{figure}[!ht]
    \centering
    \includegraphics[width=0.85\linewidth]{\Pic{png}{trisurf_2}}
    \caption{Plot loss of accuracy on the crossbar array.}
    \parbox{0.65\textwidth}{\footnotesize Note: Black lines are 1\% intervals. Red lines are 5\% intervals.}
    \label{results:fig:trisurf_2}
\end{figure}
